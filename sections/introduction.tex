\section{Introduction}

\begin{frame}{Beamer for slides}
\begin{itemize}
    \item We assume you can use \LaTeX; if you cannot,
    \hrefcol{http://en.wikibooks.org/wiki/LaTeX/}{you can learn it here}
    \item Beamer is one of the most popular and powerful document classes for presentations in \LaTeX
    \item Beamer has also a detailed
    \hrefcol{http://www.ctan.org/tex-archive/macros/latex/contrib/beamer/doc/beameruserguide.pdf}{user manual}
    \item Here we will present only the most basic features to get you up to speed
\end{itemize}
\end{frame}

%=======================================================================

\begin{frame}{Beamer vs. PowerPoint}
Compared to PowerPoint, using \LaTeX\ is better because:
\begin{itemize}
    \item It is not What-You-See-Is-What-You-Get, but What-You-\emph{Mean}-Is-What-You-Get:\\ you write the content, the computer does the typesetting
    \item Produces a \texttt{pdf}: no problems with fonts, formulas, program versions
    \item Easier to keep consistent style, fonts, highlighting, etc.
    \item Math typesetting in \TeX\ is the best:
\begin{equation*}
    \mathrm{i}\,\hslash\frac{\partial}{\partial t} \Psi(\mathbf{r},t) = -\frac{\hslash^2}{2\,m}\nabla^2\Psi(\mathbf{r},t) + V(\mathbf{r})\Psi(\mathbf{r},t)
\end{equation*}
\end{itemize}
\end{frame}

%=======================================================================

\begin{frame}[fragile]{Getting Started}
\framesubtitle{Selecting the Theme}
To start working with \texttt{beamer\_statale}, start a \LaTeX\ document with the preamble:
\begin{colorblock}[black]{statalegrey}{Minimum Statale Beamer Document}
    \verb|\documentclass{beamer}|\\
    \verb|\usetheme{statale}|\\
    \verb|\begin{document}|\\
    \verb|\begin{frame}{Hello, world!}|\\
    \verb|\end{frame}|\\
    \verb|\end{document}|\\
\end{colorblock}
\end{frame}

%=======================================================================

\begin{frame}[fragile]{Title page}
To set a typical title page, you call some commands in the preamble:
\begin{block}{The Commands for the Title Page}
\begin{verbatim}
\title{Sample Title}
\subtitle{Sample subtitle}
\author{First Author, Second Author}
\date{\today} % Can also be (ab)used for conference name &c.
\end{verbatim}
\end{block}
You can then write out the title page with \verb|\maketitle|.

To set a \textbf{background image} use the \verb|\titlebackground| command  before \verb|\maketitle|; its only argument is the name (or path) of a graphic file.

If you use the \textbf{starred version} \verb|\titlebackground*|, the image  will be clipped to a split view on the right side of the title slide.
\end{frame}

%=======================================================================

\begin{frame}[fragile]{Writing a Simple Slide}
\framesubtitle{It's really easy!}
\begin{itemize}[<+->]
    \item A typical slide has bulleted lists
    \item These can be uncovered in sequence
\end{itemize}
\begin{block}{Code for a Page with an Itemised List}<+->
\begin{verbatim}
\begin{frame}{Writing a Simple Slide}
    \framesubtitle{It's really easy!}
        \begin{itemize}[<+->]
            \item A typical slide has bulleted lists
        \item These can be uncovered in sequence
\end{itemize}\end{frame}
\end{verbatim}
\end{block}

\end{frame}